\documentclass{article}
\usepackage{amsmath}
\usepackage{amsthm}
\usepackage{tikz}
\usetikzlibrary{arrows}
\newcommand\independent{\protect\mathpalette{\protect\independenT}{\perp}}
\def\independenT#1#2{\mathrel{\rlap{$#1#2$}\mkern2mu{#1#2}}}
\newtheorem{factors}{Theorem}
\begin{document}

Following Yassin-Kassab's 1998 paper and with the three node diagram shown in 
the yEd file.

\begin{equation}
\sum_{i\in I_j} NP_{ij}q_{ij} = q_{j0}\quad \forall j \in J
\end{equation}

where $I_j$ is the set of source nodes supplying node $j$ and $J$ is
the set of all demand nodes in the network, and $q_{j0}$ is the
outflow at node $j$.

\begin{equation}
\sum_{i \in I_j} NP_{ij}q_{0i}p_{ij} = q_{j0}\quad \forall j \in J
\end{equation}

where $q_{01}$ is the inflow at node $i$ and $p_{ij}$ is the probability
that a flow at node $i$ reaches node $j$.




\begin{equation}
\frac{p_{ij}}{p_{kj}} = \frac{\alpha_i}{\alpha_k}\quad \forall i,k \in I_j; \forall j \in J
\end{equation}

Substituting this equation back into 2, we arrive at the following equation

\begin{equation}
p_{ij} = \frac{q_{j0}\alpha_i}{\sum_{i\in I_j} NP_{ij}q_{0i}\alpha_i}\quad \forall i,k \in I_j; \forall j \in J
\end{equation}

where we set $\alpha_1$ = 1.

Now we also have the condition that

\begin{equation}
\sum_{j\in J_i} NP_{ij}p_{ij} = 1\quad \forall i \in I
\end{equation}

I indexes the set of source nodes.

Subsituting the path flow probabilities into this equation we get a
set of equations which we can solve.

Taken our simple subway example

\begin{align}
\frac{20}{20 + 10\alpha_3} + \frac{10}{20 + 10\alpha_2} &= 1 \\
\frac{10\alpha_2}{10\alpha_2 + 10\alpha_3} + \frac{10\alpha_2}{20 + 10\alpha_2} &= 1 \\
\frac{10\alpha_3}{10\alpha_2 + 10\alpha_3} + \frac{20\alpha_3}{20 + 10\alpha_3} &= 1
\end{align}

When we solve for this system, we have 
$\alpha_2=1.3146$ and $\alpha_3=0.864$ so that 

\begin{align}
p_{1,9} &= \frac{20}{20 + 10\cdot 0.864} &= 0.698\\
p_{1,10} &= \frac{10}{20 + 10\cdot 1.3146} &= 0.301\\
p_{2,8} &= \frac{10\cdot 1.3146}{10\cdot 0.864 + 10\cdot 1.3146} &= 0.604\\
p_{2,10} &= \frac{10\cdot 1.3146}{20 + 10\cdot 1.3146} &= 0.396 \\
p_{3,8} &= \frac{10\cdot 0.864}{10\cdot 0.864 + 10\cdot 1.3146} &= 0.396 \\
p_{3,9} &= \frac{20\cdot 0.864}{20 + 10\cdot 0.864} &= 0.604 
\end{align}

So that we have the flow paths
\begin{align}
q_{1,9} &= 13.96 \\
q_{1,10} &= 6.04 \\
q_{2,8} &= 6.04 \\
q_{2,10} &= 3.96 \\
q_{3,8} & = 3.96 \\
q_{3,9} & = 6.04
\end{align}

So that the track segments have this flow

\begin{align}
t_{1,4} &= 20 \\
t_{4,5} &= 6.04 \\
t_{2,5} &= 3.96 \\
t_{5,10} &= 10 \\
t_{3,6} &= 10 \\
t_{6,7} &= 3.96 \\
t_{2,7} &= 6.04 \\
t_{7,8} &= 10 \\
t_{4,9} &= 13.96 \\
t_{6,9} &= 6.04
\end{align}

\end{document}